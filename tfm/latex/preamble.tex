
% MÁRGENES: 2,5 cm sup. e inf.; 3 cm izdo. y dcho.
\usepackage[
a4paper,
vmargin=2.5cm,
hmargin=3cm
]{geometry}

% INTERLINEADO: Estrecho (6 ptos./interlineado 1,15) o Moderado (6 ptos./interlineado 1,5)
\renewcommand{\baselinestretch}{1.15}
\parskip=6pt

% DEFINICIÓN DE COLORES para portada y listados de código
\usepackage[table]{xcolor}
\definecolor{azulUC3M}{RGB}{0,0,102}
\definecolor{gray97}{gray}{.97}
\definecolor{gray75}{gray}{.75}
\definecolor{gray45}{gray}{.45}

% Soporte para GENERAR PDF/A --es importante de cara a su inclusión en e-Archivo porque es el formato óptimo de preservación y a la generación de metadatos, tal y como se describe en http://uc3m.libguides.com/ld.php?content_id=31389625. 

% En la plantilla incluimos el archivo OUTPUT.XMPDATA. Puedes descargar este archivo e incluir los metadatos que se incorporarán al archivo PDF cuando compiles el archivo memoria.tex. Después vuelve a subirlo a tu proyecto. 
\usepackage[a-1b]{pdfx}

% ENLACES
\usepackage{hyperref}
\hypersetup{colorlinks=true,
	linkcolor=black, % enlaces a partes del documento (p.e. índice) en color negro
	urlcolor=blue} % enlaces a recursos fuera del documento en azul

% EXPRESIONES MATEMÁTICAS
\usepackage{amsmath,amssymb,amsfonts,amsthm}

% Codificación caracteres
\usepackage{txfonts} 
\usepackage[T1]{fontenc}
\usepackage[utf8]{inputenc}

% Definición idioma español
\usepackage[spanish, es-tabla]{babel} 
\usepackage[babel, spanish=spanish]{csquotes}
\AtBeginEnvironment{quote}{\small}

% diseño de PIE DE PÁGINA
\usepackage{fancyhdr}
\pagestyle{fancy}
\fancyhf{}
\renewcommand{\headrulewidth}{0pt}
\rfoot{\thepage}
\fancypagestyle{plain}{\pagestyle{fancy}}

% DISEÑO DE LOS TÍTULOS de las partes del trabajo (capítulos y epígrafes o subcapítulos)
\usepackage{titlesec}
\usepackage{titletoc}
\titleformat{\chapter}[block]
{\large\bfseries\filcenter}
{\thechapter.}
{5pt}
{\MakeUppercase}
{}
\titlespacing{\chapter}{0pt}{0pt}{*3}
\titlecontents{chapter}
[0pt]                                               
{}
{\contentsmargin{0pt}\thecontentslabel.\enspace\uppercase}
{\contentsmargin{0pt}\uppercase}                        
{\titlerule*[.7pc]{.}\contentspage}                 

\titleformat{\section}
{\bfseries}
{\thesection.}
{5pt}
{}
\titlecontents{section}
[5pt]                                               
{}
{\contentsmargin{0pt}\thecontentslabel.\enspace}
{\contentsmargin{0pt}}
{\titlerule*[.7pc]{.}\contentspage}

\titleformat{\subsection}
{\normalsize\bfseries}
{\thesubsection.}
{5pt}
{}
\titlecontents{subsection}
[10pt]                                               
{}
{\contentsmargin{0pt}                          
	\thecontentslabel.\enspace}
{\contentsmargin{0pt}}                        
{\titlerule*[.7pc]{.}\contentspage}  


% DISEÑO DE TABLAS
\usepackage{multirow} % permite combinar celdas 
\usepackage{caption} % para personalizar el título de tablas y figuras
\usepackage{floatrow} % utilizamos este paquete y sus macros \ttabbox y \ffigbox para alinear los nombres de tablas y figuras de acuerdo con el estilo definido.
\usepackage{array} % con este paquete podemos definir en la siguiente línea un nuevo tipo de columna para tablas: ancho personalizado y contenido centrado
\newcolumntype{P}[1]{>{\centering\arraybackslash}p{#1}}
\DeclareCaptionFormat{upper}{#1#2\uppercase{#3}\par}

% Diseño de tabla para ingeniería
\captionsetup*[table]{
	format=hang, % upper
	name=Tabla,
	justification=centering,
	labelsep=period,
	width=.75\linewidth,
	labelfont=small,
	font=small,
}

% DISEÑO DE FIGURAS. 
\usepackage{graphicx}
\usepackage{svg}
\graphicspath{{imgs/}} % ruta a la carpeta de imágenes

% Diseño de figuras para ingeniería
\captionsetup[figure]{
	format=hang,
	name=Fig.,
	singlelinecheck=off,
	labelsep=period,
	labelfont=small,
	font=small		
}

% NOTAS A PIE DE PÁGINA
\usepackage{chngcntr} % para numeración continua de las notas al pie
\counterwithout{footnote}{chapter}

% LISTADOS DE CÓDIGO
% soporte y estilo para listados de código. Más información en https://es.wikibooks.org/wiki/Manual_de_LaTeX/Listados_de_código/Listados_con_listings
\usepackage{listings}

% Mismo estilo que otros indices
\makeatletter
\let\my@chapter\@chapter
\renewcommand*{\@chapter}{%
  \addtocontents{lol}{\protect\addvspace{10pt}}%
  \my@chapter}
\makeatother

% definimos un estilo de listings
\lstdefinestyle{estilo}{
    frame=Ltb,
	framerule=0pt,
	aboveskip=0.5cm,
	framextopmargin=3pt,
	framexbottommargin=3pt,
	framexleftmargin=0.4cm,
	framesep=0pt,
	rulesep=.4pt,
	backgroundcolor=\color{gray97},
	rulesepcolor=\color{black},
	%
	basicstyle=\ttfamily\footnotesize,
	keywordstyle=\bfseries,
	stringstyle=\ttfamily,
	showstringspaces = false,
	commentstyle=\color{gray45},     
	%
	numbers=left,
	numbersep=15pt,
	numberstyle=\tiny,
	numberfirstline = false,
	breaklines=true,
	xleftmargin=\parindent
}

\captionsetup*[lstlisting]{font=small, labelsep=period} % move caption of listing

% fijamos el estilo a utilizar 
\lstset{style=estilo}
\renewcommand{\lstlistingname}{Cód.}
\renewcommand{\lstlistlistingname}{Índice de Códigos}


% DISEÑO DE GLOSARIO
\usepackage[acronym, nonumberlist]{glossaries} % glossary and acronyms
\makeglossaries

\newacronym{uav}{UAV}{Unmanned Aerial Vehicle}
\newacronym{uas}{UAS}{Unmanned Aircraft System}
\newacronym{vant}{VANT}{Vehículo Aéreo No Tripulado}
\newacronym{vslam}{vSLAM}{Visual Simultaneous Location And Mapping}
\newacronym{gps}{GPS}{Global Positioning System}
\newacronym{slam}{SLAM}{Simultaneous Location And Mapping}
\newacronym{api}{API}{Application Programming Interface}
\newacronym{uhf}{UHF}{Ultra High Frequency}
\newacronym{wifi}{WiFi}{Wireless Fidelity}
\newacronym{ieee}{IEEE}{Institute of Electrical and Electronics Engineers}
\newacronym{gcs}{GCS}{Ground Control Station}
\newacronym{cvar}{CVAR}{Computer Vision Aerial Robotics}
\newacronym{upm}{UPM}{Universidad Politecnica de Madrid}
\newacronym{ros}{ROS}{Robot Operating System}
\newacronym{yarp}{YARP}{Yet Another Robot Platform}
\newacronym{carmen}{CARMEN}{CARnegie MEllon robot Navigation}
\newacronym{orocos}{OROCOS}{Open RObot Control Software}
\newacronym{osrf}{OSRF}{Open Source Robotics Foundation}
\newacronym{3dr}{3DR}{3D Robotics}
\newacronym{dji}{DJI}{Dà-Jiāng Innovations}
\newacronym{sitl}{SITL}{Software In The Loop}
\newacronym{sdf}{SDF}{Structure Data File}
\newacronym{mavlink}{MAVLink}{Micro Air Vehicle Link}
\newacronym{lsi}{LSI}{Laboratorio de Sistemas Inteligentes}
\newacronym{uc3m}{UC3M}{Universidad Carlos III de Madrid}
\newacronym{som}{SOM}{System On Module}
\newacronym{cpu}{CPU}{Central Processing Unit}
\newacronym{gpu}{GPU}{Graphics Processing Unit}
\newacronym{sdk}{SDK}{Software Development Kit}
\newacronym{cuda}{CUDA}{Compute Unified Device Architecture}
\newacronym{usb}{USB}{Universal Serial Bus}
\newacronym{lts}{LTS}{Long Term Support}
\newacronym{mavros}{MAVROS}{MAVLink to ROS}
\newacronym{opencv}{OpenCV}{Open Computer Vision}
\newacronym{numpy}{NumPy}{Numerical Python}
\newacronym{udp}{UDP}{User Datagram Protocol}
\newacronym{yolo}{YOLO}{You Only Look Once}
\newacronym{coco}{COCO}{Common Objects in Context}
\newacronym{urjc}{URJC}{Universidad Rey Juan Carlos}
\newacronym{mlp}{MLP}{MultiLayer Perceptron}
\newacronym{ned}{NED}{North-East-Down}
\newacronym{pid}{PID}{Proporcional Integral Derivativo}
\newacronym{hsv}{HSV}{Hue Saturation Value}
\newacronym{rgb}{RGB}{Red Green Blue}
\newacronym{Fifo}{FIFO}{First In First Out}
\newacronym{rf}{RF}{Radio Frequency}
\newacronym{lidar}{LiDAR}{Light Detection and Ranging}
\newacronym{aesa}{AESA}{Agencia Estatal de Seguridad Aérea}
\newacronym{eea}{EEA}{European Economic Area}
\newacronym{sesar}{SESAR}{Single European Sky ATM Research}
\newacronym{atm}{ATM}{Air Traffic Management}
\newacronym{imu}{IMU}{Inertial Measurement Unit}

%BIBLIOGRAFÍA 

% CONFIGURACIÓN PARA LA BIBLIOGRAFÍA IEEE
\usepackage[backend=biber, style=ieee, isbn=false,sortcites, maxbibnames=6, minbibnames=1]{biblatex} % Configuración para el estilo de citas de IEEE, recomendado para el área de ingeniería. "maxbibnames" indica que a partir de 6 autores trunque la lista en el primero (minbibnames) y añada "et al." tal y como se utiliza en el estilo IEEE.

% Añadimos las siguientes indicaciones para mejorar la adaptación del estilos en español
\DefineBibliographyStrings{spanish}{%
	andothers = {et\addabbrvspace al\adddot}
}
\DefineBibliographyStrings{spanish}{
	url = {\adddot\space[En línea]\adddot\space Disponible en:}
}
\DefineBibliographyStrings{spanish}{
	urlseen = {Acceso:}
}
\DefineBibliographyStrings{spanish}{
	pages = {pp\adddot},
	page = {p.\adddot}
}

\addbibresource{refs.bib} % llama al archivo referencias.bib en el que deberá estar la bibliografía utilizada


% OTHER PKGS

\usepackage{lipsum} % Lorem ipsum, texto de prueba
\usepackage{subfiles} % Organizar el texto en diferentes ficheros
\usepackage{subcaption} % Varias imagenes en una misma figura
