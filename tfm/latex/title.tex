\begin{titlepage}
	\begin{sffamily}
	\color{azulUC3M}
	\begin{center}
		\begin{figure}[H] %incluimos el logotipo de la Universidad
			\makebox[\textwidth][c]{\includegraphics[width=16cm]{logo_UC3M.png}}
		\end{figure}
		\vspace{2.5cm}
		\begin{Large}
			Máster Universitario en Robótica y Automatización\\			
			 2021-2022\\ %Indica el curso académico
			\vspace{2cm}		
			\textsl{Trabajo Fin de Máster}
			\bigskip
			
		\end{Large}
		 	{\Huge  Infraestructura de programación de robots aéreos y aplicaciones visuales con aprendizaje profundo}\\
		 	\vspace*{0.5cm}
	 		\rule{10.5cm}{0.1mm}\\
			\vspace*{0.9cm}
			{\LARGE Pedro Arias Pérez}\\ 
			\vspace*{1cm}
		\begin{Large}
			Tutor/es\\
			David Martín Gómez \\
			José María Cañas Plaza \\
			Leganés, 11 de febrero de 2022\\
		\end{Large}
	\end{center}
	\vfill
	\color{black}
% 	\fbox{
% 		\begin{minipage}{\linewidth}
% 		\textbf{DETECCIÓN DEL PLAGIO}\\
% 		\footnotesize{La Universidad utiliza el programa \textbf{Turnitin Feedback Studio} para comparar la originalidad del trabajo entregado por cada estudiante con millones de recursos electrónicos y detecta aquellas partes del texto copiadas y pegadas. Copiar o plagiar en un TFM es considerado una \textbf{\underline{Falta Grave}}, y puede conllevar la expulsión definitiva de la Universidad.}\end{minipage}}
	
	% SI NUESTRO TRABAJO SE VA A PUBLICAR CON UNA LICENCIA CREATIVE COMMONS, INCLUIR ESTAS LÍNEAS. ES LA OPCIÓN RECOMENDADA.
	\noindent\includegraphics[width=4.2cm]{creativecommons.png}\\ %incluimos el logotipo de Creative Commons
	\footnotesize{Esta obra se encuentra sujeta a la licencia Creative Commons \textbf{Reconocimiento - No Comercial - Sin Obra Derivada}}
	
	\end{sffamily}
\end{titlepage}
