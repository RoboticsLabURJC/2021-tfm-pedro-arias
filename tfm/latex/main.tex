%----------
%   IMPORTANTE
%----------

% Esta plantilla está basada en las recomendaciones de la guía "Trabajo fin de Máster: Escribir el TFM", que encontrarás en http://uc3m.libguides.com/TFM/escribir
% contiene recomendaciones de la Biblioteca basadas principalmente en estilos APA e IEEE, pero debes seguir siempre las orientaciones de tu Tutor de TFM y la normativa de TFM para tu titulación.


% ESTA PLANTILLA ESTÁ BASADA EN EL ESTILO IEEE

%----------
%	CONFIGURACIÓN DEL DOCUMENTO
%----------

\documentclass[12pt]{report} % fuente a 12pt

% MÁRGENES: 2,5 cm sup. e inf.; 3 cm izdo. y dcho.
\usepackage[
a4paper,
vmargin=2.5cm,
hmargin=3cm
]{geometry}

% INTERLINEADO: Estrecho (6 ptos./interlineado 1,15) o Moderado (6 ptos./interlineado 1,5)
\renewcommand{\baselinestretch}{1.15}
\parskip=6pt

% DEFINICIÓN DE COLORES para portada y listados de código
\usepackage[table]{xcolor}
\definecolor{azulUC3M}{RGB}{0,0,102}
\definecolor{gray97}{gray}{.97}
\definecolor{gray75}{gray}{.75}
\definecolor{gray45}{gray}{.45}

% Soporte para GENERAR PDF/A --es importante de cara a su inclusión en e-Archivo porque es el formato óptimo de preservación y a la generación de metadatos, tal y como se describe en http://uc3m.libguides.com/ld.php?content_id=31389625. 

% En la plantilla incluimos el archivo OUTPUT.XMPDATA. Puedes descargar este archivo e incluir los metadatos que se incorporarán al archivo PDF cuando compiles el archivo memoria.tex. Después vuelve a subirlo a tu proyecto. 
\usepackage[a-1b]{pdfx}

% ENLACES
\usepackage{hyperref}
\hypersetup{colorlinks=true,
	linkcolor=black, % enlaces a partes del documento (p.e. índice) en color negro
	urlcolor=blue} % enlaces a recursos fuera del documento en azul

% EXPRESIONES MATEMÁTICAS
\usepackage{amsmath,amssymb,amsfonts,amsthm}

% Codificación caracteres
\usepackage{txfonts} 
\usepackage[T1]{fontenc}
\usepackage[utf8]{inputenc}

% Definición idioma español
\usepackage[spanish, es-tabla]{babel} 
\usepackage[babel, spanish=spanish]{csquotes}
\AtBeginEnvironment{quote}{\small}

% diseño de PIE DE PÁGINA
\usepackage{fancyhdr}
\pagestyle{fancy}
\fancyhf{}
\renewcommand{\headrulewidth}{0pt}
\rfoot{\thepage}
\fancypagestyle{plain}{\pagestyle{fancy}}

% DISEÑO DE LOS TÍTULOS de las partes del trabajo (capítulos y epígrafes o subcapítulos)
\usepackage{titlesec}
\usepackage{titletoc}
\titleformat{\chapter}[block]
{\large\bfseries\filcenter}
{\thechapter.}
{5pt}
{\MakeUppercase}
{}
\titlespacing{\chapter}{0pt}{0pt}{*3}
\titlecontents{chapter}
[0pt]                                               
{}
{\contentsmargin{0pt}\thecontentslabel.\enspace\uppercase}
{\contentsmargin{0pt}\uppercase}                        
{\titlerule*[.7pc]{.}\contentspage}                 

\titleformat{\section}
{\bfseries}
{\thesection.}
{5pt}
{}
\titlecontents{section}
[5pt]                                               
{}
{\contentsmargin{0pt}\thecontentslabel.\enspace}
{\contentsmargin{0pt}}
{\titlerule*[.7pc]{.}\contentspage}

\titleformat{\subsection}
{\normalsize\bfseries}
{\thesubsection.}
{5pt}
{}
\titlecontents{subsection}
[10pt]                                               
{}
{\contentsmargin{0pt}                          
	\thecontentslabel.\enspace}
{\contentsmargin{0pt}}                        
{\titlerule*[.7pc]{.}\contentspage}  


% DISEÑO DE TABLAS
\usepackage{multirow} % permite combinar celdas 
\usepackage{caption} % para personalizar el título de tablas y figuras
\usepackage{floatrow} % utilizamos este paquete y sus macros \ttabbox y \ffigbox para alinear los nombres de tablas y figuras de acuerdo con el estilo definido.
\usepackage{array} % con este paquete podemos definir en la siguiente línea un nuevo tipo de columna para tablas: ancho personalizado y contenido centrado
\newcolumntype{P}[1]{>{\centering\arraybackslash}p{#1}}
\DeclareCaptionFormat{upper}{#1#2\uppercase{#3}\par}

% Diseño de tabla para ingeniería
\captionsetup*[table]{
	format=hang, % upper
	name=Tabla,
	justification=centering,
	labelsep=period,
	width=.75\linewidth,
	labelfont=small,
	font=small,
}

% DISEÑO DE FIGURAS. 
\usepackage{graphicx}
\usepackage{svg}
\graphicspath{{imgs/}} % ruta a la carpeta de imágenes

% Diseño de figuras para ingeniería
\captionsetup[figure]{
	format=hang,
	name=Fig.,
	singlelinecheck=off,
	labelsep=period,
	labelfont=small,
	font=small		
}

% NOTAS A PIE DE PÁGINA
\usepackage{chngcntr} % para numeración continua de las notas al pie
\counterwithout{footnote}{chapter}

% LISTADOS DE CÓDIGO
% soporte y estilo para listados de código. Más información en https://es.wikibooks.org/wiki/Manual_de_LaTeX/Listados_de_código/Listados_con_listings
\usepackage{listings}

% Mismo estilo que otros indices
\makeatletter
\let\my@chapter\@chapter
\renewcommand*{\@chapter}{%
  \addtocontents{lol}{\protect\addvspace{10pt}}%
  \my@chapter}
\makeatother

% definimos un estilo de listings
\lstdefinestyle{estilo}{
    frame=Ltb,
	framerule=0pt,
	aboveskip=0.5cm,
	framextopmargin=3pt,
	framexbottommargin=3pt,
	framexleftmargin=0.4cm,
	framesep=0pt,
	rulesep=.4pt,
	backgroundcolor=\color{gray97},
	rulesepcolor=\color{black},
	%
	basicstyle=\ttfamily\footnotesize,
	keywordstyle=\bfseries,
	stringstyle=\ttfamily,
	showstringspaces = false,
	commentstyle=\color{gray45},     
	%
	numbers=left,
	numbersep=15pt,
	numberstyle=\tiny,
	numberfirstline = false,
	breaklines=true,
	xleftmargin=\parindent
}

\captionsetup*[lstlisting]{font=small, labelsep=period} % move caption of listing

% fijamos el estilo a utilizar 
\lstset{style=estilo}
\renewcommand{\lstlistingname}{Cód.}
\renewcommand{\lstlistlistingname}{Índice de Códigos}


% DISEÑO DE GLOSARIO
\usepackage[acronym, nonumberlist]{glossaries} % glossary and acronyms
\makeglossaries

\newacronym{uav}{UAV}{Unmanned Aerial Vehicle}
\newacronym{uas}{UAS}{Unmanned Aircraft System}
\newacronym{vant}{VANT}{Vehículo Aéreo No Tripulado}
\newacronym{vslam}{vSLAM}{Visual Simultaneous Location And Mapping}
\newacronym{gps}{GPS}{Global Positioning System}
\newacronym{slam}{SLAM}{Simultaneous Location And Mapping}
\newacronym{api}{API}{Application Programming Interface}
\newacronym{uhf}{UHF}{Ultra High Frequency}
\newacronym{wifi}{WiFi}{Wireless Fidelity}
\newacronym{ieee}{IEEE}{Institute of Electrical and Electronics Engineers}
\newacronym{gcs}{GCS}{Ground Control Station}
\newacronym{cvar}{CVAR}{Computer Vision Aerial Robotics}
\newacronym{upm}{UPM}{Universidad Politecnica de Madrid}
\newacronym{ros}{ROS}{Robot Operating System}
\newacronym{yarp}{YARP}{Yet Another Robot Platform}
\newacronym{carmen}{CARMEN}{CARnegie MEllon robot Navigation}
\newacronym{orocos}{OROCOS}{Open RObot Control Software}
\newacronym{osrf}{OSRF}{Open Source Robotics Foundation}
\newacronym{3dr}{3DR}{3D Robotics}
\newacronym{dji}{DJI}{Dà-Jiāng Innovations}
\newacronym{sitl}{SITL}{Software In The Loop}
\newacronym{sdf}{SDF}{Structure Data File}
\newacronym{mavlink}{MAVLink}{Micro Air Vehicle Link}
\newacronym{lsi}{LSI}{Laboratorio de Sistemas Inteligentes}
\newacronym{uc3m}{UC3M}{Universidad Carlos III de Madrid}
\newacronym{som}{SOM}{System On Module}
\newacronym{cpu}{CPU}{Central Processing Unit}
\newacronym{gpu}{GPU}{Graphics Processing Unit}
\newacronym{sdk}{SDK}{Software Development Kit}
\newacronym{cuda}{CUDA}{Compute Unified Device Architecture}
\newacronym{usb}{USB}{Universal Serial Bus}
\newacronym{lts}{LTS}{Long Term Support}
\newacronym{mavros}{MAVROS}{MAVLink to ROS}
\newacronym{opencv}{OpenCV}{Open Computer Vision}
\newacronym{numpy}{NumPy}{Numerical Python}
\newacronym{udp}{UDP}{User Datagram Protocol}
\newacronym{yolo}{YOLO}{You Only Look Once}
\newacronym{coco}{COCO}{Common Objects in Context}
\newacronym{urjc}{URJC}{Universidad Rey Juan Carlos}
\newacronym{mlp}{MLP}{MultiLayer Perceptron}
\newacronym{ned}{NED}{North-East-Down}
\newacronym{pid}{PID}{Proporcional Integral Derivativo}
\newacronym{hsv}{HSV}{Hue Saturation Value}
\newacronym{rgb}{RGB}{Red Green Blue}
\newacronym{Fifo}{FIFO}{First In First Out}
\newacronym{rf}{RF}{Radio Frequency}
\newacronym{lidar}{LiDAR}{Light Detection and Ranging}
\newacronym{aesa}{AESA}{Agencia Estatal de Seguridad Aérea}
\newacronym{eea}{EEA}{European Economic Area}
\newacronym{sesar}{SESAR}{Single European Sky ATM Research}
\newacronym{atm}{ATM}{Air Traffic Management}
\newacronym{imu}{IMU}{Inertial Measurement Unit}

%BIBLIOGRAFÍA 

% CONFIGURACIÓN PARA LA BIBLIOGRAFÍA IEEE
\usepackage[backend=biber, style=ieee, isbn=false,sortcites, maxbibnames=6, minbibnames=1]{biblatex} % Configuración para el estilo de citas de IEEE, recomendado para el área de ingeniería. "maxbibnames" indica que a partir de 6 autores trunque la lista en el primero (minbibnames) y añada "et al." tal y como se utiliza en el estilo IEEE.

% Añadimos las siguientes indicaciones para mejorar la adaptación del estilos en español
\DefineBibliographyStrings{spanish}{%
	andothers = {et\addabbrvspace al\adddot}
}
\DefineBibliographyStrings{spanish}{
	url = {\adddot\space[En línea]\adddot\space Disponible en:}
}
\DefineBibliographyStrings{spanish}{
	urlseen = {Acceso:}
}
\DefineBibliographyStrings{spanish}{
	pages = {pp\adddot},
	page = {p.\adddot}
}

\addbibresource{refs.bib} % llama al archivo referencias.bib en el que deberá estar la bibliografía utilizada


% OTHER PKGS

\usepackage{lipsum} % Lorem ipsum, texto de prueba
\usepackage{subfiles} % Organizar el texto en diferentes ficheros
\usepackage{subcaption} % Varias imagenes en una misma figura


%-------------
%	DOCUMENTO
%-------------

\begin{document}
\pagenumbering{roman} % Se utilizan cifras romanas en la numeración de las páginas previas al cuerpo del trabajo
	
%----------
%	PORTADA
%----------	
\begin{titlepage}
	\begin{sffamily}
	\color{azulUC3M}
	\begin{center}
		\begin{figure}[H] %incluimos el logotipo de la Universidad
			\makebox[\textwidth][c]{\includegraphics[width=16cm]{logo_UC3M.png}}
		\end{figure}
		\vspace{2.5cm}
		\begin{Large}
			Máster Universitario en Robótica y Automatización\\			
			 2021-2022\\ %Indica el curso académico
			\vspace{2cm}		
			\textsl{Trabajo Fin de Máster}
			\bigskip
			
		\end{Large}
		 	{\Huge  Infraestructura de programación de robots aéreos y aplicaciones visuales con aprendizaje profundo}\\
		 	\vspace*{0.5cm}
	 		\rule{10.5cm}{0.1mm}\\
			\vspace*{0.9cm}
			{\LARGE Pedro Arias Pérez}\\ 
			\vspace*{1cm}
		\begin{Large}
			Tutor/es\\
			David Martín Gómez \\
			José María Cañas Plaza \\
			Leganés, 11 de febrero de 2022\\
		\end{Large}
	\end{center}
	\vfill
	\color{black}
% 	\fbox{
% 		\begin{minipage}{\linewidth}
% 		\textbf{DETECCIÓN DEL PLAGIO}\\
% 		\footnotesize{La Universidad utiliza el programa \textbf{Turnitin Feedback Studio} para comparar la originalidad del trabajo entregado por cada estudiante con millones de recursos electrónicos y detecta aquellas partes del texto copiadas y pegadas. Copiar o plagiar en un TFM es considerado una \textbf{\underline{Falta Grave}}, y puede conllevar la expulsión definitiva de la Universidad.}\end{minipage}}
	
	% SI NUESTRO TRABAJO SE VA A PUBLICAR CON UNA LICENCIA CREATIVE COMMONS, INCLUIR ESTAS LÍNEAS. ES LA OPCIÓN RECOMENDADA.
	\noindent\includegraphics[width=4.2cm]{creativecommons.png}\\ %incluimos el logotipo de Creative Commons
	\footnotesize{Esta obra se encuentra sujeta a la licencia Creative Commons \textbf{Reconocimiento - No Comercial - Sin Obra Derivada}}
	
	\end{sffamily}
\end{titlepage}

\clearpage  %página en blanco o de cortesía
\thispagestyle{empty}
\mbox{}

%----------
%	DEDICATORIA
%----------	
\clearpage  % pág dch
\vspace*{\stretch{2}}
\begin{flushright}
    \textit{Á miña familia\\ polos seus consellos,\\ que son sempre bos.}
\end{flushright}
\vspace*{\stretch{3}}
\clearpage
\thispagestyle{empty}
\mbox{}

%----------
%	RESUMEN Y PALABRAS CLAVE
%----------	
\clearpage  % pág dch
\renewcommand\abstractname{\large\bfseries\filcenter\uppercase{Resumen}}
\begin{abstract}
\thispagestyle{plain}
\setcounter{page}{3}
	
	Los robots aéreos forman parte de nuestro día a día con aplicaciones en muchos ámbitos de nuestra vida. En muchas ocasiones desarrollar estas aplicaciones puede resultar una tarea ardua o incluso imposible para usuarios con conocimientos limitados en la robótica aérea. Este trabajo busca proporcionar una infraestructura de programación de robots aéreos que facilite este tipo procesos.
	
	La infraestructura, \emph{DroneWrapper}, ofrece la posibilidad al usuario de desarrollar aplicaciones abstrayéndose de las complejidades asociadas a la aeronave a través de una sencilla interfaz de programación de usuario. La aplicación se ha implementado con ROS y Python siguiendo un diseño modular que facilita el acople de diversos \emph{drivers} que permiten extender las funcionalidades de la infraestructura. \\
	Junto con la infraestructura se han desarrollado varios \emph{drivers}. Uno de ellos es \emph{Tello Driver}, el cual permite adaptar la aplicación para su uso con los drone DJI Tello. 
	
	En paralelo, se han desarrollado dos aplicaciones que ejemplifican el uso de la infraestructura creada, \emph{sigue-color} y \emph{sigue-persona}. Ambas aplicaciones utilizan técnicas de visión por ordenador, clásicas o modernas (aprendizaje profundo), para seguir un determinado tipo de objeto. \\ Dichas aplicaciones se han probado sobre diferentes plataformas aéreas, reales y simuladas, para dar validez a la solución ofrecida. En concreto, las aeronaves utilizadas son tres; un PX4 simulado, un DJI Tello y un PX4 de construcción propia.
	
	\textbf{Palabras clave:}
	 infraestructura de programación, drones, robótica aérea, aplicaciones, visión por ordenador, aprendizaje profundo
	
	\vfill
\end{abstract}

\newpage % página en blanco o de cortesía
\thispagestyle{empty}
\mbox{}

%----------
%	AGRADECIMIENTOS
%----------	

\chapter*{Agradecimientos} % \chapter* evita que aparezca en el índice

\setcounter{page}{5}
	
	Antes de comenzar con el trabajo, creo que es justo pararme y agradecer a todas aquellas personas que me han acompañado a lo largo de este viaje y que habéis hecho posible que esté escribiendo estas palabras. Ha sido un trayecto apasionante en el que me he topado con grandes profesores, compañeros y amigos. Por ello, me gustaría, en primer lugar, dar las gracias al Laboratorio de Sistemas Inteligentes, a la Escuela Politécnica Superior, a la Universidad Carlos III de Madrid y a todos sus profesionales por su dedicación y atención ofrecida.

Además, en particular, quisiera dar las gracias a ciertas personas que me han allanado el camino, y han conseguido hacerme disfrutar de esta travesía. 

A José María, co-tutor de este trabajo. Gracias por estos últimos dos años y medio en los que nos hemos reunido una o dos veces por semana. Gracias por tu tiempo y tu paciencia, siempre intento aprender al máximo de tu conocimiento. Gracias también, por invitarme a JdeRobot, donde se ha gestado la semilla de este trabajo.

A David, co-tutor de este trabajo. Gracias por ponerlo todo tan fácil desde el inicio, por permitirme elegir la idea sobre la que desarrollar el trabajo y por facilitarme todos los medios para conseguirlo. Pero gracias sobretodo por tu tiempo, he aprendido muchas cosas estos meses escuchándote.

A mi hermana Raquel, a mis padres y a mi familia. Gracias por todo vuestro amor y apoyo, y gracias por todos vuestros consejos, el tiempo siempre os acaba dando la razón. Este trabajo es para vosotros, espero que estéis orgullosos.

A mis grandes amigos, Dani, Verto, Sunil, Juan, Ana y Carme, sois un lujo que no merezco. También a mis otros grandes amigos, Alejandro, Castell, Luci, Silvi, Pablo, Javis, Pau y Carmen, cada vez es más difícil juntarnos pero gracias por seguir ahí. 

A todos mis compañeros del máster y del laboratorio, gracias por ayudarme a superar esta etapa. En especial a los integrantes de Yarpiz; Meira, Carlos, Roca, Sergi, Aitor, Quispe y Marru. Gracias por todas las risas entre trabajos y partidos de volley.

\begin{center}
    ¡Mil gracias a todos!
\end{center}
		
	\vfill
	
	\newpage % página en blanco o de cortesía
	\thispagestyle{empty}
	\mbox{}
	
%----------
%	ÍNDICES
%----------	

%--
% Índice general
%-
\tableofcontents
\thispagestyle{fancy}

%\newpage % página en blanco o de cortesía
%\thispagestyle{empty}
%\mbox{}

%--
% Índice de figuras. Si no se incluyen, comenta las líneas siguientes
%-
\listoffigures
\thispagestyle{fancy}

%\newpage % página en blanco o de cortesía
%\thispagestyle{empty}
%\mbox{}

%--
% Índice de tablas. Si no se incluyen, comenta las líneas siguientes
%-
\listoftables
\thispagestyle{fancy}

\newpage % página en blanco o de cortesía
\thispagestyle{empty}
\mbox{}

%--
% Índice de codigos. Si no se incluyen, comenta las líneas siguientes
%-
\lstlistoflistings
\thispagestyle{fancy}

\newpage % página en blanco o de cortesía
\thispagestyle{empty}
\mbox{}

%-----------
%   Glosario
%-----------
\printglossary[type=\acronymtype, title=\uppercase{Glosario}, style=superheader]
\thispagestyle{fancy}

%\newpage % página en blanco o de cortesía
%\thispagestyle{empty}
%\mbox{}


%----------
%	MEMORIA
%----------	
\clearpage
\pagenumbering{arabic} % numeración con números arábigos para el resto de la memoria.	

\subfile{sections/01-intro} \clearpage
\subfile{sections/02-herram} \clearpage
\subfile{sections/03-metodo} \clearpage
\subfile{sections/04-infra} \clearpage
\subfile{sections/05-aplic} \clearpage
\subfile{sections/06-concl} \clearpage

%----------
%	BIBLIOGRAFÍA
%----------	

%\nocite{*} % Si quieres que aparezcan en la bibliografía todos los documentos que la componen (también los que no estén citados en el texto) descomenta está línea

\clearpage
\addcontentsline{toc}{chapter}{Bibliografía}
% \setquotestyle[english]{british} % Cambiamos el tipo de cita porque en el estilo IEEE se usan las comillas inglesas.
\printbibliography


%----------
%	ANEXOS
%----------

% Si tu trabajo incluye anexos, puedes descomentar las siguientes líneas
%\chapter* {Anexo x}
%\pagenumbering{gobble} % Las páginas de los anexos no se numeran


%------------
%   GLOSSARY
%------------

\glsaddall % show all

\end{document}