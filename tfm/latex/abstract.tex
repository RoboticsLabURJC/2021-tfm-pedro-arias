\begin{abstract}
\thispagestyle{plain}
\setcounter{page}{3}
	
	Los robots aéreos forman parte de nuestro día a día con aplicaciones en muchos ámbitos de nuestra vida. En muchas ocasiones desarrollar estas aplicaciones puede resultar una tarea ardua o incluso imposible para usuarios con conocimientos limitados en la robótica aérea. Este trabajo busca proporcionar una infraestructura de programación de robots aéreos que facilite este tipo procesos.
	
	La infraestructura, \emph{DroneWrapper}, ofrece la posibilidad al usuario de desarrollar aplicaciones abstrayéndose de las complejidades asociadas a la aeronave a través de una sencilla interfaz de programación de usuario. La aplicación se ha implementado con ROS y Python siguiendo un diseño modular que facilita el acople de diversos \emph{drivers} que permiten extender las funcionalidades de la infraestructura. \\
	Junto con la infraestructura se han desarrollado varios \emph{drivers}. Uno de ellos es \emph{Tello Driver}, el cual permite adaptar la aplicación para su uso con los drone DJI Tello. 
	
	En paralelo, se han desarrollado dos aplicaciones que ejemplifican el uso de la infraestructura creada, \emph{sigue-color} y \emph{sigue-persona}. Ambas aplicaciones utilizan técnicas de visión por ordenador, clásicas o modernas (aprendizaje profundo), para seguir un determinado tipo de objeto. \\ Dichas aplicaciones se han probado sobre diferentes plataformas aéreas, reales y simuladas, para dar validez a la solución ofrecida. En concreto, las aeronaves utilizadas son tres; un PX4 simulado, un DJI Tello y un PX4 de construcción propia.
	
	\textbf{Palabras clave:}
	 infraestructura de programación, drones, robótica aérea, aplicaciones, visión por ordenador, aprendizaje profundo
	
	\vfill
\end{abstract}